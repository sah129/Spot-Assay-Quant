% Options for packages loaded elsewhere
\PassOptionsToPackage{unicode}{hyperref}
\PassOptionsToPackage{hyphens}{url}
%
\documentclass[
]{article}
\usepackage{lmodern}
\usepackage{amssymb,amsmath}
\usepackage{ifxetex,ifluatex}
\ifnum 0\ifxetex 1\fi\ifluatex 1\fi=0 % if pdftex
  \usepackage[T1]{fontenc}
  \usepackage[utf8]{inputenc}
  \usepackage{textcomp} % provide euro and other symbols
\else % if luatex or xetex
  \usepackage{unicode-math}
  \defaultfontfeatures{Scale=MatchLowercase}
  \defaultfontfeatures[\rmfamily]{Ligatures=TeX,Scale=1}
\fi
% Use upquote if available, for straight quotes in verbatim environments
\IfFileExists{upquote.sty}{\usepackage{upquote}}{}
\IfFileExists{microtype.sty}{% use microtype if available
  \usepackage[]{microtype}
  \UseMicrotypeSet[protrusion]{basicmath} % disable protrusion for tt fonts
}{}
\makeatletter
\@ifundefined{KOMAClassName}{% if non-KOMA class
  \IfFileExists{parskip.sty}{%
    \usepackage{parskip}
  }{% else
    \setlength{\parindent}{0pt}
    \setlength{\parskip}{6pt plus 2pt minus 1pt}}
}{% if KOMA class
  \KOMAoptions{parskip=half}}
\makeatother
\usepackage{xcolor}
\IfFileExists{xurl.sty}{\usepackage{xurl}}{} % add URL line breaks if available
\IfFileExists{bookmark.sty}{\usepackage{bookmark}}{\usepackage{hyperref}}
\hypersetup{
  pdftitle={SpotAssayQuant Tutorial},
  pdfauthor={Sarah Hawbaker},
  hidelinks,
  pdfcreator={LaTeX via pandoc}}
\urlstyle{same} % disable monospaced font for URLs
\usepackage[margin=1in]{geometry}
\usepackage{longtable,booktabs}
% Correct order of tables after \paragraph or \subparagraph
\usepackage{etoolbox}
\makeatletter
\patchcmd\longtable{\par}{\if@noskipsec\mbox{}\fi\par}{}{}
\makeatother
% Allow footnotes in longtable head/foot
\IfFileExists{footnotehyper.sty}{\usepackage{footnotehyper}}{\usepackage{footnote}}
\makesavenoteenv{longtable}
\usepackage{graphicx,grffile}
\makeatletter
\def\maxwidth{\ifdim\Gin@nat@width>\linewidth\linewidth\else\Gin@nat@width\fi}
\def\maxheight{\ifdim\Gin@nat@height>\textheight\textheight\else\Gin@nat@height\fi}
\makeatother
% Scale images if necessary, so that they will not overflow the page
% margins by default, and it is still possible to overwrite the defaults
% using explicit options in \includegraphics[width, height, ...]{}
\setkeys{Gin}{width=\maxwidth,height=\maxheight,keepaspectratio}
% Set default figure placement to htbp
\makeatletter
\def\fps@figure{htbp}
\makeatother
\setlength{\emergencystretch}{3em} % prevent overfull lines
\providecommand{\tightlist}{%
  \setlength{\itemsep}{0pt}\setlength{\parskip}{0pt}}
\setcounter{secnumdepth}{-\maxdimen} % remove section numbering

\title{SpotAssayQuant Tutorial}
\author{Sarah Hawbaker}
\date{}

\begin{document}
\maketitle

{
\setcounter{tocdepth}{2}
\tableofcontents
}
\hypertarget{overview}{%
\subsection{Overview}\label{overview}}

This pipeline batch processes spot assay image sets and outputs all
result data to .csv files. In addition, it outputs a series of
supplementary data and result images for spot checking. You may find
that most of this data is unecessary; it was made with a particularly
problematic and inconsistent dataset in mind. I have chosen to keep the
auxillary data in for robust testing and verification. See the
\texttt{Input\ and\ Output} section for more information.

\hypertarget{installation-and-setup}{%
\subsection{Installation and Setup}\label{installation-and-setup}}

\hypertarget{installation}{%
\subsubsection{Installing Cellprofiler}\label{installation}}

CellProfiler is available here: \url{https://cellprofiler.org/}. Click
\textbf{Downloads \textgreater\textgreater{} CellProfiler}. Make sure to
download the basic version and not CellProfiler Analyst.

\includegraphics{Tutorial_files/figure-latex/unnamed-chunk-2-1.pdf}

\hypertarget{choosing}{%
\subsubsection{Choosing the Pipeline}\label{choosing}}

Pipelines are located in the \texttt{Pipelines} folder. There are two
pipelines available, depending on the format of your input images. For
grayscale images, use the GrayScaleInput pipeline. However, you may find
that you want more control over the thresholding values and method. In
that case, there is a BinaryInput pipeline available. Note that it is
not difficult to modify the \texttt{Threshold} module in the
GrayScaleInput pipeline, I have just created these two options for
ease-of-use.

For sample thresholding macros in ImageJ, see the \texttt{FIJI\ Macros}
folder.

\hypertarget{pipeline-setup}{%
\subsection{Pipeline Setup}\label{pipeline-setup}}

\begin{enumerate}
\def\labelenumi{\arabic{enumi}.}
\tightlist
\item
  \textbf{Open CellProfiler}. Upon opening CellProfiler, you will see a
  black terminal window pop up and run a few lines of code. This is just
  the .exe file launching. Do not close this window while you are
  running the program, or it will exit. I think it's best just to
  minimize it to avoid accidental termination.
\end{enumerate}

\includegraphics{Tutorial_files/figure-latex/unnamed-chunk-3-1.pdf}

Close the introduction window.

\begin{enumerate}
\def\labelenumi{\arabic{enumi}.}
\setcounter{enumi}{1}
\tightlist
\item
  \textbf{Load the pipeline}. Download your chosen pipeline from the
  \texttt{Pipelines} folder into a local directory. Load into
  CellProfiler via \textbf{File \textgreater\textgreater{} Import
  \textgreater\textgreater{} Pipeline from File}. Your screen should
  look like this:
\end{enumerate}

\includegraphics{Tutorial_files/figure-latex/unnamed-chunk-4-1.pdf}

\hypertarget{basic-navigation}{%
\subsubsection{Basic Navigation}\label{basic-navigation}}

The pipeline consists of a series of modules which each apply a
different function to the image. The module list is outlined in red
below, and a single module in blue.

\includegraphics{Tutorial_files/figure-latex/unnamed-chunk-5-1.pdf}

\includegraphics{Tutorial_files/figure-latex/unnamed-chunk-6-1.pdf}

\hypertarget{input-and-output}{%
\subsubsection{Input and Output}\label{input-and-output}}

Before we run the pipeline, we must make sure the output all goes to the
correct place. The output is currently structured as follows:

\begin{itemize}
\item
  Spreadseets: This is where all the raw data is stored.
\item
  Outlines: This is a visual representation of the detected spots on
  each image.
\item
  Object Areas: A visual respresentation of measured data drawn on the
  spots themselves. Currently this is set to ``Area'' but it can be
  changed in the \texttt{DisplayDataOnImage} module.
\item
  Object Numbers: Similar to the Object Areas folder, but with the
  individual spot numbers displayed on each spot. This is useful to
  verify that the rows are being grouped correctly.
\item
  Rows: A visualization of the row groupings.
\item
  Graphs: A place to store any user-generated graphs post-processing.
  See the \texttt{Post\ Processing} section for more information.
\end{itemize}

Should you not want this extra output, you can simply uncheck its
associated module. You do not need to worry about setting the output
folder if the module is unchecked.

\begin{longtable}[]{@{}lll@{}}
\toprule
Output & Associated.Module & File.Format\tabularnewline
\midrule
\endhead
Spreadsheets & ExportToSpreadsheet & .csv\tabularnewline
Outlines & SaveImages 1 & .png\tabularnewline
Object Areas & SaveImages 2 & .png\tabularnewline
Object Numbers & SaveImages 3 & .png\tabularnewline
Rows & SaveImages 4 & .png\tabularnewline
Graphs & N/A & N/A\tabularnewline
\bottomrule
\end{longtable}

\hypertarget{setting-up-the-output-folder}{%
\subsubsection{Setting up the Output
Folder:}\label{setting-up-the-output-folder}}

\begin{enumerate}
\def\labelenumi{\arabic{enumi}.}
\item
  Click \textbf{View output settings}, which is located beneath the
  module list. Set the \textbf{Default Output Folder} to your desired
  folder. You do not need to change the \textbf{Output file format}
  setting. If you want all your data to go into one folder, unsorted,
  you can skip the next step. I don't recommend this as with large
  imagesets it will be messy to sort things out afterwards.
\item
  For each of the associated modules, find the \textbf{Output File
  Location} setting. Set to your desired output folder. I like to create
  all empty folders ahead of time inside my main output folder. There is
  an example output folder HERE for reference.
\end{enumerate}

If you structure your output this way, select \textbf{Default Output
Folder sub-folder} from the drop-down menu and choose your folder from
there. Do this for each ``SaveImages'' module and the ``Export to
Spreadsheet'' module.

\includegraphics{Tutorial_files/figure-latex/unnamed-chunk-8-1.pdf}

\hypertarget{adding-input-files}{%
\subsubsection{Adding Input Files}\label{adding-input-files}}

\begin{enumerate}
\def\labelenumi{\arabic{enumi}.}
\tightlist
\item
  Click on the \textbf{Images} module. Drag and drop your files or
  folders into the pipeline. To clear all files from the pipeline, go to
  \textbf{Edit \textgreater\textgreater{} Clear File List}.
\end{enumerate}

\hypertarget{defining-the-grid}{%
\subsubsection{Defining the Grid}\label{defining-the-grid}}

Before you run the pipline you will need to set up the ``Grid'' that
determines how many rows and columns of spots you want to quantify. Set
this by clicking the \textbf{Define Grid} module and entering your
values in the \textbf{Number of rows} and **Number of columns* fields.
The default settings are 7 and 4, respectively.

If you are sure that the image size and spot placements will be
identical for each image, you can change the \textbf{Define a grid for
which cycle?} setting to \textbf{Once}. Otherwise it should stay at
\textbf{Each Cycle}.

\hypertarget{running-the-pipeline}{%
\subsection{Running the Pipeline}\label{running-the-pipeline}}

\begin{enumerate}
\def\labelenumi{\arabic{enumi}.}
\tightlist
\item
  Click \textbf{Analyze Images}. The pipeline will start. Typically it
  will take a moment or two to start up, but after that it will run
  relatively fast. As it begins to process images, you will see windows
  pop depending on which modules are set to visible.
\end{enumerate}

\hypertarget{define-grid}{%
\subsubsection{Define Grid}\label{define-grid}}

This is the first of two points in the pipeline where manual
intervention is required. If you set the \textbf{Define a grid for which
cycle?} setting to \textbf{Once} you will only need to do this step one.
Otherwise it will have to be done for each image.

Click to set the coordinates for the top-left spot and the bottom-right
spot. Keep setting until you feel comfortable that the grid is nearly
correct. We can edit later, so it does not need to be perfect.

\includegraphics{Tutorial_files/figure-latex/unnamed-chunk-9-1.pdf}

\hypertarget{edit-objects}{%
\subsubsection{Edit Objects}\label{edit-objects}}

This is the second manual intervention. This module allows you to see
the detected objects and edit them as needed. Most images should not
need any editing at all. If you find that more than 10\% need manual
editing then either the settings are incorrect or this pipeline is not
compatible with your dataset.

The edit module has a ``Show help'' which has detailed instructions for
how to use it. I will give a abridged version here, but I recommend to
use the help module at first.

Note that you cannot edit if the \texttt{pan} or \texttt{zoom} button is
selected. Always make sure to deselect after using.

The main commands you will need are:

\begin{itemize}
\item
  Toggle. Press \textbf{T} to toggle the display between full colors and
  outlines. This will make the objects easier to see and edit.
\item
  Edit mode: \textbf{Right click} an object to go into edit mode. This
  will transform the object outline to a series of control points that
  you can then edit with the commands below.
\item
  Delete: \textbf{X} to delete a series of points by selection,
  \textbf{D} to delete a single point after clicking it.
\item
  Add: \textbf{A} to add a control point at the location of your mouse
  after a single click.
\item
  Join: \textbf{J} to join two areas of control points. Even if not
  contiguous the areas will be measured and counted as one object.
\item
  Split: \textbf{S} to split an object into two. After pressing
  \textbf{S} you will click the two points that will serve as the
  dividing line.
\end{itemize}

\includegraphics{Tutorial_files/figure-latex/unnamed-chunk-10-1.pdf}

\hypertarget{post-processing}{%
\subsection{Post Processing}\label{post-processing}}

When the pipeline is completed you will see this pop-up:

\includegraphics{Tutorial_files/figure-latex/unnamed-chunk-11-1.pdf}

Go to the \textbf{Default Output Folder} to view your results.

The data is summarized and compiled using a python script. If you are
familiar with Python you can run the script. It is available as a
Jupyter Notebook file and can be found in the \texttt{Scripts} folder.
Otherwise, I am working on hosting the tool on the lab website for easy
use. Until then, email me you ``Spreadsheets'' folder and I'll run the
script for you.

\end{document}
